The regular chain classes in BPAS  provide a collection of routines
for solving systems of algebraic equations by means
of exact methods, a.k.a. symbolic computation.
The main commands for accomplishing this are the
\texttt{triangularize} and
\texttt{intersect} methods of the
\texttt{RegularChain} class and the
\texttt{intersect} method of the
\texttt{ZeroDimensionalRegularChain} class. The objects of both these
classes are \emph{regular chains}. Becuase regular chains are
mathematical objects that algebraically encode geometric components of
the solution space, the solutions to a system of algebraic equations
can be expressed as a set of regular chains. This is precisely what
\texttt{triangularize} and
\texttt{intersect} accomplish: for an input
polynomial $p$ (for \texttt{intersect}) or algebraic system $S$ (for
\texttt{triangularize}), the output is a description of the solution
set as a collection of \texttt{RegularChain} objects.
