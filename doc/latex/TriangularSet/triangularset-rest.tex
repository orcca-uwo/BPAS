The \texttt{TriangularSet} class consists of a variety of support
functions needed to manipulate triangular sets, as well as routines
designed to reduce polynomials with respect to a triangular set. In
particular, the \texttt{normalForm} routine computes the normal form
(in the sense of lexicographical Gr\"{o}bner bases, see
{\small{\url{https://en.wikipedia.org/wiki/Gr%C3%B6bner_basis}}) of a
  polynomial with respect to a triangular set when the triangular set
  is strongly normalized (the initials of all of the polynomials are
  constant), and the \texttt{pseudoDivide} routine computes the
  pseudo-remainder of a polynomial with respect to the triangular
  set.

The most important basic feature of \texttt{TriangularSet} objects is
that they come in two distinct types: fixed (intended for handling
algebraic systems) and variable (intended for handling
differential-algebraic systems). For fixed \texttt{TriangularSet}
objects, the list of variables of the \texttt{TriangularSet} is fixed
to be a certain ordered list of variables that cannot change over the
life of the object. This is when the ambient space one is working in
(explained below) is known and fixed. For variable
\texttt{TriangularSet} objects, whenever a polynomial is added to the
triangular set, any new variables in the polynomial are added to the
list of variables of the \texttt{TriangularSet}. Both fixed and
variable \texttt{TriangularSet} objects can be created with a list of
transcendental variables, \emph{i.e.}, variables that can never be
subject to algebraic constraints. In both cases, the list of
transcendental variables is fixed for the life of the object.

Objects of the \texttt{TriangularSet} class keep track of which
variables $v$ in the list of variables are subject to constraints,
\emph{i.e.}, for which there is a polynomial in the set with main
variable $v$. Such variables are called \emph{algebraic variables},
and are maintained in a special ordered list by the class (in the same
order as the list of variables). To keep this distinction between the
different kinds of variables clear, the list of variables are
described in the documentation as ``potentially algebraic'', because
they are the variables that can be subject to algebraic
constraints. Thus, the intersection of the (potentially algebraic)
variables and the transendental variables is always empty, and the
algebraic variables form a subset of the (potentially algebraic)
variables. It is the (potentially algebraic) variables that determine
the dimension of the ambient space of the triangular set, \emph{i.e.},
the (affine) space over which the polynomials of the triangular set
act as constraints. The dimension of (the saturated ideal of) the
triangular set is then the number of (potentially algebraic) variables
minus the number of algebraic variables.
