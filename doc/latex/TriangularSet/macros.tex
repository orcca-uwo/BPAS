\newcommand{\euclideanDivision}[4]{\mbox{{$ \begin{array}{c|c} #1 & #2 \\ \cline{2-2}
                                                               #3 & #4 \\ \end{array} $}}}
\newtheorem{Notation}{Notation}
\newtheorem{Hypothesis}{Hypothesis}
\newtheorem{thm}{Theorem}
\newtheorem{prop}{Proposition}
\newtheorem{defn}{Definition}
\newtheorem{lem}{Lemma}
\newtheorem{cor}{Corollary}
\newtheorem{Conjecture}{Conjecture}
\newtheorem{Remark}{Remark}
\newtheorem{Example}{Example}
\newtheorem{Algorithm}{Algorithm}
\newcommand{\myproof}{\textsc{Proof.} }
\newcommand{\myfoorp}{\hfill$\square$}
\newenvironment{Proof}{\par\noindent{\bf Proof. }}{\unskip~$\Box$\par}

\usepackage{amssymb}
\newcommand*{\QEDA}{\hfill\ensuremath{\blacksquare}}%
\newcommand*{\QEDB}{\hfill\ensuremath{\square}}%


\pagestyle{empty}

%%%%%%%%%%%%
%% Numbers %
%%%%%%%%%%%%
\def\C {\ensuremath{\mathbb{C}}}
\def\K {\ensuremath{\mathbb{K}}}
\def\Q {\ensuremath{\mathbb{Q}}}
\def\R {\ensuremath{\mathbb{R}}}
\def\Z {\ensuremath{\mathbb{Z}}}
\def\KK {\ensuremath{\mathbb{K}}}
\def\KKK {\ensuremath{\overline{\mathbb{K}}}}

%%%%%%%%%%%%%%%%%%%%%%%%%%%%%%%%%%%
%% Polynomials and regular chains %
%%%%%%%%%%%%%%%%%%%%%%%%%%%%%%%%%%%
\newcommand{\discrim}[1]{\mbox{{\rm discrim}$(#1)$}}
\newcommand{\init}[1]{\mbox{{\rm init}$(#1)$}}
\newcommand{\iter}[1]{\mbox{{\rm iter}$(#1)$}}
\newcommand{\mdeg}[1]{\mbox{{\rm mdeg}$(#1)$}}
\newcommand{\mvar}[1]{\mbox{{\rm mvar}$(#1)$}}
\newcommand{\prem}[1]{\mbox{{\rm prem}$(#1)$}}
\newcommand{\pquo}[1]{\mbox{{\rm pquo}$(#1)$}}
\newcommand{\rank}[1]{\mbox{{\rm rank}$(#1)$}}
\newcommand{\ires}[1]{\mbox{{\rm ires}$(#1)$}}
\newcommand{\src}[1]{\mbox{{\rm src}$(#1)$}}
\newcommand{\sat}[1]{\mbox{{\rm sat}$(#1)$}}
\newcommand{\sep}[1]{\mbox{{\rm sep}$(#1)$}}
\newcommand{\tail}[1]{\mbox{{\rm tail}$(#1)$}}
\newcommand{\head}[1]{\mbox{{\rm head}$(#1)$}}

%%%%%%%%%%%%%%%%%%%%%%
%% Software packages %
%%%%%%%%%%%%%%%%%%%%%%
\newcommand{\RegularChains}{{\tt RegularChains}}
\newcommand{\Maple}{\textsc{Maple}}
\newcommand{\bpas}{\textsc{bpas}}
\newcommand{\cumodp}{\textsc{cumodp}}
\newcommand{\cilkplus}{\textsc{CilkPlus}}
\newcommand{\mpsolve}{\textsc{MPSolve}}
\newcommand{\matlab}{\textsc{Matlab}}

%%%%%%%%%%%%%%%%%%%%%%
%% From preamble.tex %
%%%%%%%%%%%%%%%%%%%%%%


% FANCY ENUMERATION
%% This gives us fun enumeration environments. compactenum will be nice.
\usepackage{paralist}
% An itemize-style list with lots of space between items
\newenvironment{outerlist}[1][\enskip\textbullet]%
        {\begin{enumerate}[#1]}{\end{enumerate}%
         \vspace{-.6\baselineskip}}
% An itemize-style list with little space between items
\newenvironment{innerlist}[1][\enskip\textbullet]%
        {\begin{compactenum}[#1]}{\end{compactenum}}

\renewcommand{\ie}{\emph{i.e.}}
\renewcommand{\eg}{\emph{e.g.}}
\renewcommand{\etc}{\emph{etc.}}
\newcommand{\nn}{\nonumber}

\newcommand{\bt}{\begin{thm}}
\newcommand{\et}{\end{thm}}
\newcommand{\bco}{\begin{cor}}
\newcommand{\eco}{\end{cor}}
\newcommand{\bl}{\begin{lem}}
\newcommand{\el}{\end{lem}}
\newcommand{\bp}{\begin{prop}}
\newcommand{\ep}{\end{prop}}
\newcommand{\bdf}{\begin{defn}}
\newcommand{\edf}{\end{defn}}
\newcommand{\bpf}{\begin{proof}}
\newcommand{\epf}{\end{proof}}
\newcommand{\bcor}{\begin{correct}}
\newcommand{\ecor}{\end{correct}}
\newcommand{\bci}{\begin{compactitem}}
\newcommand{\eci}{\end{compactitem}}

\newcommand{\be}{\begin{enumerate}}
\newcommand{\ee}{\end{enumerate}}
\newcommand{\bi}{\begin{itemize}}
\newcommand{\ei}{\end{itemize}}
\newcommand{\bd}{\begin{description}}
\newcommand{\ed}{\end{description}}
\newcommand{\bc}{\begin{center}}
\newcommand{\ec}{\end{center}}
\newcommand{\bq}{\begin{quote}\small}
\newcommand{\eq}{\end{quote}}
\newcommand{\beq}{\begin{equation}}
\newcommand{\eeq}{\end{equation}}
\newcommand{\bea}{\setlength\arraycolsep{2pt}\begin{eqnarray}}
\newcommand{\eea}{\end{eqnarray}\setlength\arraycolsep{6pt}}
\newcommand{\bfig}{\begin{figure}}
\newcommand{\efig}{\end{figure}}
\newcommand{\bcom}{\begin{comment}}
\newcommand{\ecom}{\end{comment}}
\newcommand{\etal}{\emph{et.\ al.}}
\newcommand{\viz}{\emph{viz.}}
\newcommand{\cf}{\emph{cf.}}
\newcommand{\p}{\paragraph{}}
\newcommand{\s}{\section}
\newcommand{\subs}{\subsection}
\newcommand{\subss}{\subsubsection}
\newcommand{\st}{\section*}
\newcommand{\subst}{\subsection*}
\newcommand{\subsst}{\subsubsection*}

%%%%%%%%%%%%%%%%%%%%%%%%%%%
% For pseudocode and code %
%%%%%%%%%%%%%%%%%%%%%%%%%%%

\usepackage{color}
\usepackage{listings}
\lstset{ %
language=C++,                % choose the language of the code
basicstyle=\footnotesize,       % the size of the fonts that are used for the code
numbers=left,                   % where to put the line-numbers
numberstyle=\footnotesize,      % the size of the fonts that are used for the line-numbers
stepnumber=1,                   % the step between two line-numbers. If it is 1 each line will be numbered
numbersep=5pt,                  % how far the line-numbers are from the code
backgroundcolor=\color{white},  % choose the background color. You must add \usepackage{color}
showspaces=false,               % show spaces adding particular underscores
showstringspaces=false,         % underline spaces within strings
showtabs=false,                 % show tabs within strings adding particular underscores
frame=single,           % adds a frame around the code
tabsize=2,          % sets default tabsize to 2 spaces
captionpos=b,           % sets the caption-position to bottom
breaklines=true,        % sets automatic line breaking
breakatwhitespace=false,    % sets if automatic breaks should only happen at whitespace
escapeinside={\%*}{*)}          % if you want to add a comment within your code
}
\lstset{language=C++}
\usepackage{algpseudocode}
\usepackage{algorithm}
\newcommand{\alnam}[1]{\textsc{#1}}
\newcommand{\templ}[1]{$\langle$#1$\rangle$}
\newcommand{\vect}[1]{\boldsymbol{#1}}

%%%%%%%%%%%%%%%%%%%%%%%%%%%
%%% Miscellaneous macros %%%
%%%%%%%%%%%%%%%%%%%%%%%%%%%%
\newcommand{\hidetext}[1]{\mbox{ \ }}
\newcommand{\todo}[2]{{{\textcolor{red}{ #1}}}\footnote{ {\textcolor{blue}{ #2}} }}
\newcommand{\todolater}[2]{#1}
\newcommand{\fix}[3]{#1}
%% \newcommand{\reworked}[1]{{{\textcolor{blue}{ #1}}}}
\newcommand{\reworked}[1]{{ #1}}
